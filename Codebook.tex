\documentclass[a4paper,10pt,twocolumn,oneside]{article}
\setlength{\columnsep}{10pt}                                                                    %兩欄模式的間距
\setlength{\columnseprule}{0pt}                                                                %兩欄模式間格線粗細

\usepackage{amsthm}								%定義,例題
\usepackage{amssymb}
%\usepackage[margin=2cm]{geometry}
\usepackage{fontspec}								%設定字體
\usepackage{color}
\usepackage[x11names]{xcolor}
\usepackage{listings}								%顯示code用的
%\usepackage[Glenn]{fncychap}						%排版,頁面模板
\usepackage{fancyhdr}								%設定頁首頁尾
\usepackage{graphicx}								%Graphic
\usepackage{enumerate}
\usepackage{multicol}
\usepackage{titlesec}
\usepackage{amsmath}
\usepackage[CheckSingle, CJKmath]{xeCJK}
\usepackage{savetrees}
\usepackage{array}
\usepackage{xparse}
% \usepackage{CJKulem}

%\usepackage[T1]{fontenc}
\usepackage{amsmath, courier, listings, fancyhdr, graphicx}
\topmargin=0pt
\headsep=5pt
\textheight=780pt
\footskip=0pt
\voffset=-40pt
\textwidth=545pt
\marginparsep=0pt
\marginparwidth=0pt
\marginparpush=0pt
\oddsidemargin=0pt
\evensidemargin=0pt
\hoffset=-42pt

\titlespacing\section{0pt}{-2pt plus 0pt minus 2pt}{-1pt plus 0pt minus 2pt}
\titlespacing\subsection{0pt}{-2pt plus 0pt minus 2pt}{-1pt plus 0pt minus 2pt}
\titlespacing\subsubsection{0pt}{-2pt plus 0pt minus 2pt}{-1pt plus 0pt minus 2pt}


%\renewcommand\listfigurename{圖目錄}
%\renewcommand\listtablename{表目錄} 

%%%%%%%%%%%%%%%%%%%%%%%%%%%%%

\setmainfont{Ubuntu}				%主要字型
\setmonofont{Ubuntu Mono}
\XeTeXlinebreaklocale "zh"						%中文自動換行
\XeTeXlinebreakskip = 0pt plus 1pt				%設定段落之間的距離
\setcounter{secnumdepth}{3}						%目錄顯示第三層

%%%%%%%%%%%%%%%%%%%%%%%%%%%%%
\newcommand\digitstyle{\color{DarkOrchid3}}
\makeatletter
\lst@CCPutMacro\lst@ProcessOther {"2D}{\lst@ttfamily{-{}}{-{}}}
\@empty\z@\@empty

\newtoks\BBQube@token
\newcount\BBQube@length
\def\BBQube@ResetToken{\BBQube@token{}\BBQube@length\z@}
\def\BBQube@Append#1{\advance\BBQube@length\@ne
  \BBQube@token=\expandafter{\the\BBQube@token#1}}

\def\BBQube@ProcessChar#1{%
  \ifnum\lst@mode=\lst@Pmode%
    \ifnum 9<1#1%
      \expandafter\BBQube@Append{\begingroup\digitstyle #1 \endgroup}%
    \else%
      \expandafter\BBQube@Append{#1}%
    \fi%
  \else%
    \expandafter\BBQube@Append{#1}%
  \fi%
}
\def\BBQube@ProcessStringInner#1#2\BBQube@nil{%
  \expandafter\BBQube@ProcessChar{#1}%
  \if\relax\detokenize{#2}\relax%
  \else%
    \expandafter\BBQube@ProcessStringInner#2\BBQube@nil%
  \fi%
}

\def\BBQube@ProcessString#1{\expandafter\BBQube@ProcessStringInner#1\BBQube@nil}

\lst@AddToHook{OutputOther}{%
\BBQube@ResetToken%
\expandafter\BBQube@ProcessString{\the\lst@token}%
\lst@token=\expandafter{\the\BBQube@token}%
}
\makeatother
\lstset{											% Code顯示
language=C++,										% the language of the code
basicstyle=\footnotesize\ttfamily, 						% the size of the fonts that are used for the code
numbers=left,										% where to put the line-numbers
numberstyle=\tiny,						% the size of the fonts that are used for the line-numbers
stepnumber=1,										% the step between two line-numbers. If it's 1, each line  will be numbered
numbersep=5pt,										% how far the line-numbers are from the code
backgroundcolor=\color{white},					% choose the background color. You must add \usepackage{color}
showspaces=false,									% show spaces adding particular underscores
showstringspaces=false,							% underline spaces within strings
showtabs=false,									% show tabs within strings adding particular underscores
frame=false,											% adds a frame around the code
tabsize=2,											% sets default tabsize to 2 spaces
captionpos=b,										% sets the caption-position to bottom
breaklines=true,									% sets automatic line breaking
breakatwhitespace=true,							% sets if automatic breaks should only happen at whitespace
escapeinside={\%*}{*)},							% if you want to add a comment within your code
morekeywords={constexpr},									% if you want to add more keywords to the set
keywordstyle=\bfseries\color{Blue1},
commentstyle=\itshape\color{Red4},
stringstyle=\itshape\color{Green4},
}

%%%%%%%%%%%%%%%%%%%%%%%%%%%%%

\ExplSyntaxOn
\NewDocumentCommand{\captureshell}{som}
 {
  \sdaau_captureshell:Ne \l__sdaau_captureshell_out_tl { #3 }
  \IfBooleanT { #1 }
   {% we may need to stringify the result
    \tl_set:Nx \l__sdaau_captureshell_out_tl
     { \tl_to_str:N \l__sdaau_captureshell_out_tl }
   }
  \IfNoValueTF { #2 }
   {
    \tl_use:N \l__sdaau_captureshell_out_tl
   }
   {
    \tl_set_eq:NN #2 \l__sdaau_captureshell_out_tl
   }
 }

\tl_new:N \l__sdaau_captureshell_out_tl

\cs_new_protected:Nn \sdaau_captureshell:Nn
 {
  \sys_get_shell:nnN { #2 } { } #1
  \tl_trim_spaces:N #1 % remove leading and trailing spaces
 }
\cs_generate_variant:Nn \sdaau_captureshell:Nn { Ne }
\ExplSyntaxOff

\begin{document}
\pagestyle{fancy}
\fancyfoot{}
%\fancyfoot[R]{\includegraphics[width=20pt]{ironwood.jpg}}
\fancyhead[L]{National Tsing Hua University XL-pants}
\fancyhead[R]{\thepage}
\renewcommand{\headrulewidth}{0.4pt}
\renewcommand{\contentsname}{Contents} 
\newcommand{\inputcode}[2]{
    \subsection[#1]{#1 \footnotesize{[\texttt{\captureshell{cpp #2 -dD -P -fpreprocessed | tr -d '[:space:]' | md5sum | cut -c-6}}]}}
    \lstinputlisting{#2}
}

\newcommand{\includepy}[2]{ % 支援 Python (.py)
  \subsection{#1}
    \lstinputlisting[language=python]{#2}
}

\newcommand{\includetex}[2]{ % 支援 Latex (.tex)
  \subsection{#1}
    \input{#2}
}

\textbf{
\scriptsize
\begin{multicols}{2}
  \tableofcontents
\end{multicols}
}
%%%%%%%%%%%%%%%%%%%%%%%%%%%%%

%\newpage

\footnotesize
\section{Basic}
  \inputcode{Default code}{./codes/1_Basic/Default_code.cpp}
% \inputcode{Shell script}{./codes/1_Basic/Shell_script.cpp}
  \inputcode{Pragma}{./codes/1_Basic/Pragma.cpp}
  \inputcode{readchar}{./codes/1_Basic/readchar.cpp}
  \inputcode{debug}{./codes/1_Basic/debug.cpp}
  \inputcode{vimrc}{./codes/1_Basic/vimrc.cpp}
%  \inputcode{Texas holdem}{./codes/1_Basic/Texas_holdem.cpp}
  \inputcode{black magic}{./codes/1_Basic/black_magic.cpp}
\section{Graph}
  \inputcode{SCC}{./codes/2_Graph/SCC.cpp}
  \inputcode{Bridge}{./codes/2_Graph/Bridge.cpp}
  \inputcode{BCC Vertex}{./codes/2_Graph/BCC_Vertex.cpp}
  \inputcode{Dominator Tree}{./codes/2_Graph/Dominator_Tree.cpp}
  \inputcode{2SAT}{./codes/2_Graph/2SAT.cpp}
  \inputcode{MinimumMeanCycle}{./codes/2_Graph/MinimumMeanCycle.cpp}
  \inputcode{Virtual Tree}{./codes/2_Graph/Virtual_Tree.cpp}
  \inputcode{Maximum Clique Dyn}{./codes/2_Graph/Maximum_Clique_Dyn.cpp}
%  \inputcode{Minimum Arborescence fast}{./codes/2_Graph/Minimum_Arborescence_fast.cpp}
  \inputcode{NumberofMaximalClique}{./codes/2_Graph/NumberofMaximalClique.cpp}
%  \inputcode{Vizing}{./codes/2_Graph/Vizing.cpp}
  \inputcode{MinimumSteinerTree}{./codes/2_Graph/MinimumSteinerTree.cpp}
  \inputcode{Minimum Arborescence}{./codes/2_Graph/Minimum_Arborescence.cpp}
  \inputcode{Maximum Clique}{./codes/2_Graph/Maximum_Clique.cpp}
  \inputcode{Minimum Clique Cover}{./codes/2_Graph/Minimum_Clique_Cover.cpp}
  \inputcode{Is Planar}{./codes/2_Graph/Is_Planar.cpp}
\section{Data Structure}
  %\inputcode{2D Segment Tree}{./codes/3_Data_Structure/2D_Segment_Tree.cpp}
  \inputcode{Sparse table}{./codes/3_Data_Structure/Sparse_table.cpp}
  \inputcode{Binary Index Tree}{./codes/3_Data_Structure/Binary_Index_Tree.cpp}
  \inputcode{Segment Tree}{./codes/3_Data_Structure/Segment_Tree.cpp}
  \inputcode{BIT kth}{./codes/3_Data_Structure/BIT_kth.cpp}
%  \inputcode{DSU}{./codes/3_Data_Structure/DSU.cpp}
  \inputcode{Centroid Decomposition}{./codes/3_Data_Structure/Centroid_Decomposition.cpp}
%  \inputcode{Smart Pointer}{./codes/3_Data_Structure/Smart_Pointer.cpp}
  \inputcode{IntervalContainer}{./codes/3_Data_Structure/IntervalContainer.cpp}
%  \inputcode{KDTree useful}{./codes/3_Data_Structure/KDTree_useful.cpp}
  \inputcode{KDTree}{./codes/3_Data_Structure/KDTree.cpp}
  \inputcode{min heap}{./codes/3_Data_Structure/min_heap.cpp}
  \inputcode{LiChaoST}{./codes/3_Data_Structure/LiChaoST.cpp}
  \inputcode{Treap}{./codes/3_Data_Structure/Treap.cpp}
  \inputcode{link cut tree}{./codes/3_Data_Structure/link_cut_tree.cpp}
  \inputcode{Heavy light Decomposition}{./codes/3_Data_Structure/Heavy_light_Decomposition.cpp}
%  \inputcode{Leftist Tree}{./codes/3_Data_Structure/Leftist_Tree.cpp}
  \inputcode{Range Chmin Chmax Add Range Sum}{./codes/3_Data_Structure/Range_Chmin_Chmax_Add_Range_Sum.cpp}
  \inputcode{discrete trick}{./codes/3_Data_Structure/discrete_trick.cpp}
\section{Flow Matching}
  \includetex{Model}{./codes/4_Flow_Matching/Model.tex}
  \inputcode{Dinic}{./codes/4_Flow_Matching/Dinic.cpp}
  \inputcode{Maximum Simple Graph Matching}{./codes/4_Flow_Matching/Maximum_Simple_Graph_Matching.cpp}
  \inputcode{Kuhn Munkres}{./codes/4_Flow_Matching/Kuhn_Munkres.cpp}
%  \inputcode{MincostMaxflow dijkstra}{./codes/4_Flow_Matching/MincostMaxflow_dijkstra.cpp}
  \inputcode{General Matching Random}{./codes/4_Flow_Matching/Genaral_Matching_Random.cpp}
  \inputcode{isap}{./codes/4_Flow_Matching/isap.cpp}
  \inputcode{Gomory Hu tree}{./codes/4_Flow_Matching/Gomory_Hu_tree.cpp}
  \inputcode{MincostMaxflow}{./codes/4_Flow_Matching/MincostMaxflow.cpp}
  \inputcode{SW-mincut}{./codes/4_Flow_Matching/SW-mincut.cpp}
%  \inputcode{Maximum Weight Matching}{./codes/4_Flow_Matching/Maximum_Weight_Matching.cpp}
%  \inputcode{Minimum Weight Matching wrong}{./codes/4_Flow_Matching/Minimum_Weight_Matching_wrong.cpp}
  \inputcode{Bipartite Matching}{./codes/4_Flow_Matching/Bipartite_Matching.cpp}
  \inputcode{BoundedFlow}{./codes/4_Flow_Matching/BoundedFlow.cpp}
%  \inputcode{MinCostCirculation}{./codes/4_Flow_Matching/MinCostCirculation.cpp}
\section{String}
  \inputcode{Smallest Rotation}{./codes/5_String/Smallest_Rotation.cpp}
  \inputcode{KMP}{./codes/5_String/KMP.cpp}
  \inputcode{Manacher}{./codes/5_String/Manacher.cpp}
  \inputcode{De Bruijn sequence}{./codes/5_String/De_Bruijn_sequence.cpp}
  \inputcode{SAM}{./codes/5_String/SAM.cpp}
  \inputcode{Aho-Corasick Automatan}{./codes/5_String/Aho-Corasick_Automatan.cpp}
%  \inputcode{SAIS-old}{./codes/5_String/SAIS-old.cpp}
  \inputcode{Z-value}{./codes/5_String/Z-value.cpp}
  \inputcode{exSAM}{./codes/5_String/exSAM.cpp}
%  \inputcode{SAIS}{./codes/5_String/SAIS.cpp}
  \inputcode{SAIS-C++20}{./codes/5_String/SAIS-C++20.cpp}
%  \inputcode{PalTree}{./codes/5_String/PalTree.cpp}
  \inputcode{MainLorentz}{./codes/5_String/MainLorentz.cpp}
  \inputcode{Suffix Array}{./codes/5_String/Suffix_Array.cpp}
\section{Math}
  \includetex{numbers}{./codes/6_Math/numbers.tex}
  \includetex{Estimation}{./codes/6_Math/Estimation.tex}
  \inputcode{chineseRemainder}{./codes/6_Math/chineseRemainder.cpp}
  \inputcode{Pirime Count}{./codes/6_Math/PiCount.cpp}
  \inputcode{floor sum}{./codes/6_Math/floor_sum.cpp}
  \inputcode{QuadraticResidue}{./codes/6_Math/QuadraticResidue.cpp}
  \inputcode{floor enumeration}{./codes/6_Math/floor_enumeration.cpp}
  \inputcode{ax+by=gcd}{./codes/6_Math/ax+by=gcd.cpp}
  \inputcode{cantor expansion}{./codes/6_Math/cantor_expansion.cpp}
  \includetex{Generating function}{./codes/6_Math/Generating_function.tex}
  \inputcode{Fraction}{./codes/6_Math/Fraction.cpp}
  \inputcode{Gaussian gcd}{./codes/6_Math/Gaussian_gcd.cpp}
  \includetex{Theorem}{./codes/6_Math/Theorem.tex}
  \inputcode{Determinant}{./codes/6_Math/Determinant.cpp}
  \inputcode{ModMin}{./codes/6_Math/ModMin.cpp}
  \inputcode{Simultaneous Equations}{./codes/6_Math/Simultaneous_Equations.cpp}
  \inputcode{Big number}{./codes/6_Math/Big_number.cpp}
  \includetex{Euclidean}{./codes/6_Math/Euclidean.tex}
  \inputcode{Primes}{./codes/6_Math/Primes.cpp}
  \inputcode{Miller Rabin}{./codes/6_Math/Miller_Rabin.cpp}
  \inputcode{Pollard Rho}{./codes/6_Math/Pollard_Rho.cpp}
  \inputcode{Berlekamp-Massey}{./codes/6_Math/Berlekamp-Massey.cpp}
  \inputcode{floor ceil}{./codes/6_Math/floor_ceil.cpp}
  \inputcode{fac no p}{./codes/6_Math/fac_no_p.cpp}
  \inputcode{DiscreteLog}{./codes/6_Math/DiscreteLog.cpp}
  \includetex{SimplexConstruction}{./codes/6_Math/SimplexConstruction.tex}
  \inputcode{Simplex Algorithm}{./codes/6_Math/Simplex_Algorithm.cpp}
  \inputcode{SchreierSims}{./codes/6_Math/SchreierSims.cpp}
\section{Polynomial}
  %\inputcode{Polynomial Operation}{./codes/7_Polynomial/Polynomial_Operation.cpp}
  \inputcode{Fast Walsh Transform}{./codes/7_Polynomial/Fast_Walsh_Transform.cpp}
  \inputcode{NTT.2}{./codes/7_Polynomial/NTT.2.cpp}
  \inputcode{Number Theory Transform}{./codes/7_Polynomial/Number_Theory_Transform.cpp}
  \inputcode{Fast Fourier Transform}{./codes/7_Polynomial/Fast_Fourier_Transform.cpp}
  \inputcode{Value Poly}{./codes/7_Polynomial/Value_Poly.cpp}
  \includetex{Newton}{./codes/7_Polynomial/Newton.tex}
\section{Geometry}
  \inputcode{Default code}{./codes/8_Geometry/Default_code.cpp}
  \inputcode{Default code int}{./codes/8_Geometry/Default_code_int.cpp}
  \inputcode{Convex hull}{./codes/8_Geometry/Convex_hull.cpp}
  \inputcode{PointInConvex}{./codes/8_Geometry/PointInConvex.cpp}
  \inputcode{PolyUnion}{./codes/8_Geometry/PolyUnion.cpp}
  \inputcode{external bisector}{./codes/8_Geometry/external_bisector.cpp}
  \inputcode{Convexhull3D}{./codes/8_Geometry/Convexhull3D.cpp}
  \inputcode{Triangulation Vonoroi}{./codes/8_Geometry/Triangulation_Vonoroi.cpp}
  \inputcode{Polar Angle Sort}{./codes/8_Geometry/Polar_Angle_Sort.cpp}
  \inputcode{Intersection of polygon and circle}{./codes/8_Geometry/Intersection_of_polygon_and_circle.cpp}
  \inputcode{Tangent line of two circles}{./codes/8_Geometry/Tangent_line_of_two_circles.cpp}
  \inputcode{CircleCover}{./codes/8_Geometry/CircleCover.cpp}
  \inputcode{Heart}{./codes/8_Geometry/Heart.cpp}
  \inputcode{PointSegDist}{./codes/8_Geometry/PointSegDist.cpp}
  \inputcode{Minkowski Sum}{./codes/8_Geometry/Minkowski_Sum.cpp}
  \inputcode{TangentPointToHull}{./codes/8_Geometry/TangentPointToHull.cpp}
  \inputcode{Intersection of two circles}{./codes/8_Geometry/Intersection_of_two_circles.cpp}
  \inputcode{Intersection of line and circle}{./codes/8_Geometry/Intersection_of_line_and_circle.cpp}
%  \inputcode{Trapezoidalization}{./codes/8_Geometry/Trapezoidalization.cpp}
  \inputcode{point in circle}{./codes/8_Geometry/point_in_circle.cpp}
  \inputcode{PolyCut}{./codes/8_Geometry/PolyCut.cpp}
  \inputcode{minDistOfTwoConvex}{./codes/8_Geometry/minDistOfTwoConvex.cpp}
%  \inputcode{DelaunayTriangulation}{./codes/8_Geometry/DelaunayTriangulation.cpp}
  \inputcode{rotatingSweepLine}{./codes/8_Geometry/rotatingSweepLine.cpp}
  \inputcode{Intersection of line and convex}{./codes/8_Geometry/Intersection_of_line_and_convex.cpp}
  \inputcode{3Dpoint}{./codes/8_Geometry/3Dpoint.cpp}
  \inputcode{HPIGeneralLine}{./codes/8_Geometry/HPIGeneralLine.cpp}
  \inputcode{minMaxEnclosingRectangle}{./codes/8_Geometry/minMaxEnclosingRectangle.cpp}
  \inputcode{Half plane intersection}{./codes/8_Geometry/Half_plane_intersection.cpp}
  \inputcode{Vector in poly}{./codes/8_Geometry/Vector_in_poly.cpp}
%  \inputcode{DelaunayTriangulation dq}{./codes/8_Geometry/DelaunayTriangulation_dq.cpp}
  \inputcode{Minimum Enclosing Circle}{./codes/8_Geometry/Minimum_Enclosing_Circle.cpp}
\section{Else}
  \inputcode{ManhattanMST}{./codes/9_Else/ManhattanMST.cpp}
  \inputcode{Mos Algorithm With modification}{./codes/9_Else/Mos_Algorithm_With_modification.cpp}
  \inputcode{BitsetLCS}{./codes/9_Else/BitsetLCS.cpp}
  \inputcode{BinarySearchOnFraction}{./codes/9_Else/BinarySearchOnFraction.cpp}
  \inputcode{SubsetSum}{./codes/9_Else/SubsetSum.cpp}
  \inputcode{DynamicConvexTrick}{./codes/9_Else/DynamicConvexTrick.cpp}
  \inputcode{DynamicMST}{./codes/9_Else/DynamicMST.cpp}
  \includetex{Matroid}{./codes/9_Else/Matroid.tex}
%  \inputcode{cyclicLCS}{./codes/9_Else/cyclicLCS.cpp}
  \inputcode{HilbertCurve}{./codes/9_Else/HilbertCurve.cpp}
  \inputcode{Mos Algorithm On Tree}{./codes/9_Else/Mos_Algorithm_On_Tree.cpp}
  \includetex{Mos Algorithm}{./codes/9_Else/Mos_Algorithm.tex}
  \inputcode{AdaptiveSimpson}{./codes/9_Else/AdaptiveSimpson.cpp}
  \inputcode{min plus convolution}{./codes/9_Else/min_plus_convolution.cpp}
  \inputcode{cyc tsearch}{./codes/9_Else/cyc_tsearch.cpp}
  \inputcode{All LCS}{./codes/9_Else/All_LCS.cpp}
  \inputcode{NQueens}{./codes/9_Else/NQueens.cpp}
  \inputcode{simulated annealing}{./codes/9_Else/simulated_annealing.cpp}
  \inputcode{DLX}{./codes/9_Else/DLX.cpp}
  \inputcode{tree hash}{./codes/9_Else/tree_hash.cpp}
  \inputcode{tree knapsack}{./codes/9_Else/tree_knapsack.cpp}
%  \inputcode{DynamicConvexTrick bb}{./codes/9_Else/DynamicConvexTrick_bb.cpp}
%\section{JAVA}
%  \inputcode{Big number}{./codes/10_JAVA/Big_number.cpp}
%\section{Python}
%  \includepy{misc}{./codes/11_Python/misc.py}


\end{document}
